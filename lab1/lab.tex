\documentclass{ffslides}
\ffpage{32}{\numexpr 16/9}
\usepackage{documentation,fancyvrb}
\begin{document}

\blankpage \btext{.435}{.45}{.13}{
 CS 124 Lab 1 \\ By: Amuldeep Dhillon
}
\blankpage\dtext{Dictionary program}
\btext{.1}{.3}{.5}{
This program works as an online dictionary.  It reads a word
    and shows its translation.  The program loads the dictionary
    data from the ``dict.dat'' file. (Visible on next page.)
}
\ctext{.1}{.4}{.75}{
{\color{blue} The objective of the lab is to review the programming techniques learned
in cs102. \\ The sections in parenthesis suggest where in ``C++ for You++'' to review.}\\
\qi{Building programs from source code ({\bf compile and link})(1.5)}
\qi{Communicating with the user using the console ({\bf cin} and {\bf cout})(2.10)}
\qi{Creating and using {\bf const} for data that will not change as program is running(3.6)}
\qi{Using boolean values(6.3)}
\qi{Reading data from a file into an array (or vector)(7.8)}
\qi{Using the 3 control structures: {\bf sequence, selection, and iteration}(9.1)}
\qi{Searching an array (or vector) for a value(9.3)}
\qi{Using {\bf pass-by-reference} in order to have a function ``return'' more than one value(11.3)}
\qi{Using {\bf std::string} and {\bf std::vector}(Appendix B, and 21.1)}
\qii{The texbook shows us how to create these ourselves, but will use the built-in versions at first}
\qi{Create documentation using \LaTeX(not in textbook)}
{\color{blue} Chapter 2 may be enough to get you started, but as soon and as much as you can, you should read
all of {\it Part One : chapters 1-13}. The focus of this semester is {\it Part Two : chapters 14-28}.
}}

\blankpage\dtext{Dictionary program}
\btext{.1}{.3}{.5}{
This program works as an online dictionary.  It reads a word
    and shows its translation.  The program loads the dictionary
    data from the ``dict.dat'' file.
} 
\ctext{.1}{.4}{.5}{
\VerbatimInput{dict.dat}
}


\blankpage\dtext{
lab.h
\inputsourcecode{lab.h}
}
\btext{.5}{.2}{.3}{
Contains all the header files and shared functions. Entry is a struct 
made to hold the dictionary info. So that we can store all the entries
in a single array.
}
\blankpage\dtext{
lab.cpp
\inputsourcecode{main.cpp}
}
\btext{.5}{.35}{.23}{
\qi{Load the dictionary from the file}
\qi{Translate words}
}

\blankpage\dtext{
loadDictionary.cpp
\inputsourcecode{loadDictionary.cpp}
}
\btext{.5}{.15}{.33}{
\qi{Reads dictionary entries from a file.}
\qi{Returns true if successful, false if cannot open the file.}
\qi{Open dictionary file}
\qi{Read and display the header line}
\qi{Read words and translations into the dictionary array}
\qi{Report the number of entries}
}
     

\blankpage\dtext{
foundWord.cpp
\inputsourcecode{foundWord.cpp}
}
\btext{.3}{.3}{.2}{
\qi{Searches dictionary for word}
\qii{If found it returns true}
\qii{If not found it returns false}
}

\blankpage \dtext{
TEST of Program}

\putfig{.07}{.15}{.6}{lab}

\btext{.7}{.2}{.25}{
\qi{The program turns Dog to Kootha}
\qi{The program turns Cat to Billi}
\qi{The program turns I to Mai}
\qi{The program closes for words it does not know such as "We"}
}

\end{document}

